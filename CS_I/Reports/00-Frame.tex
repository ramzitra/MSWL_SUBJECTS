\documentclass[11pt]{article}

\title{\textbf{CASE STUDIES I: Translations in Libre Software}}
\author{Sergio Arroutbi Braojos}
\date{\today}
\usepackage{listings}
\addtolength{\voffset}{-50pt}
\begin{document}

\maketitle

\section{Introduction}
Translations in FLOSS mean a very important part for its knowledge, development and spread. However, it is often a forgiven issue inside the projects, above all those which are starting and are still far from maturity.\\
\\
When talking about translations environment in Libre software, some concepts must be clarified:
\begin{enumerate}
\item Internationalisation (i18n): Capability that software has to be localizable with no further concerns.
\item Localisation (l10n): The action to perform adaptation of a certain software or free software project related issue to a certain language, usually named \emph{locale}
\item Translation: Adaptation of a text to a certain language. It is normally used for non-code issues, such as documentation.
\end{enumerate}
In terms of FLOSS and free cultural works, its free nature means translations fit well into both open models, as:
\begin{itemize}
\item Free software allows derived work, so translations can and must be acomplished
\item Non code parts (e.g., wikies or documentation), are normally free cultural works, so translation fit as well
\item Besides this, internationalisation, localisation and translation allow Open Source to reach to further people and a wider bunch of scenarios, so they become a MUST
\end{itemize}
\section{Localisation process}
\subsection{Objects part of localisation}
The first important part for a Software Project is identifying those objects that are likely to be internationalized. Some examples could be:
\begin{itemize}
\item{Messages in the code, to dump to interface, log, to send data to other users, etc.}
\item{Help, Documentation}
\item{Installation programs}
\item{Administration interfaces}
\item{Marketing issues}
\end{itemize}
Depending on the activity of a project, people involved and community behind the project, the localisation is more or less mature. For example, KDE contains applications and documentation localised, while Debian, also carries it out on installation, web, wiki and specific packages.
\subsection{Internationalisation}
In order to achieve internationalitation, two main issues have to be considered.
On the one hand, internationalisation guides and \textbf{i18n + l10n mailing lists}, help. On the other hand, there are some file formats and conventions that can be used, from just plain text files, through PO or XLIFF formats, having to do with Gettext and OAXAL respectively.\\
\\
Apart from that, there is a variety of \textbf{tools and platforms} available. Among them, it is worth to distinguish:
\begin{itemize}
\item{GNU Gettext}
\item{Android l10n guide}
\item{KDE POT scripts}
\item{GNOME Damned lies}
\item{Web l10n platforms}
\end{itemize}
\subsection{Localisation}
Meanwhile, in terms of localisation, next tools have to be highlighted:
\begin{itemize}
\item{Poedit}
\item{Gtranslator}
\item{Lokalize}
\item{Virtaal}
\end{itemize}
Apart from them, there is a bunch of translation web platforms to consider:
\begin{itemize}
\item{Pootle}
\item{Launchpad}
\item{Transifex}
\item{Weblite}
\end{itemize}

\subsection{Review, Maintenance, Quality Assurance}
Importance of localisation has already been defined. Accomplishment of a good localisation process is based, basically, on the next steps:
\begin{enumerate}
\item{\textbf{Translators are part of the community}, and should be taken into account in the project strategies, as well as in release schedule}.
\item{Software and non-software developments should contemplate translations and its issues, in all possible environments}.
\item{Authorsip of translations have to be tracked, as well as credited. Responsabilities have to be defined too}.
\end{enumerate}
\subsection{Benefits and disadvantages}
To conclude, a list of benefits and disadvantages is proposed. \textbf{Among benefits}, \textbf{increasing market}, extra features, and extra recruitment have to be contemplated, as well as \textbf{new business opportunities} and social benefits.\\
\\
Meanwhile, \textbf{as disadvantages}, localisation means \textbf{design challenges}. Moreover this, the \textbf{coordination process} related to developers and translators \textbf{increases its difficulty}. Apart from that, there is a work \textbf{overload} affecting the community, in terms of crowdsourcing and Quality Assurance.
\end{document}
