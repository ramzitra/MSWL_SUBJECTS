\documentclass[10pt, a4paper, oneside]{report}
\usepackage[utf8]{inputenc}

\begin{document}
\title{The Libre Software Forgotten Names}
\author{Sergio Arroutbi}
\date{12 October 2012}

\section* {Introduction}

Richard Stallman, Linus Torvalds, Donald Ervin Knuth, Bruce Perens, Eric Schmidt, Miguel de Icaza, Mattias Ettrich, ... There is a bunch of recognized names in Free Software world.\\ 
Meanwhile, on the other hand, there is a much bigger bunch of names that are not as recognized as they should.\\
The aim of this text is to bring those names back to memory, in order to recognize their contribution to the Open Source Community, obviously, from a very personal point of view.

\section* {Applications}

Software Applications is, in all probability, the most productive scope of the Libre Software movement. In fact, the main idea of the movement is around software itself, although other technology fields have incorporated the idea as well, as Linux Distributions or Programming Languages theirselves.
The following list includes software developers that have been great contributors to Open Source Community:\\
\\
\textbf{\underline{Brian J. Fox:}}\\
Active developer of the FSF. Author of GNU Bash, GNU Makeinfo, GNU Info, GNU Finger and readline libraries.\\
\\
\textbf{\underline{Alan Cox:}}\\
One of the first Linux kernel instalators, Cox discovered and fixed many of existing bugs and went on to rewrite much of the networking subsystem. 
He then became one of the main developers and maintainers of the whole kernel.
He has also been involved in the GNOME and X.Org projects.\\
\\
\textbf{\underline{Mark Spencer:}}\\
Computer engineer, original author of the GTK instant messaging client Gaim (renamed to Pidgin).
He is also the creator of Asterisk, a Linux-based open-sourced PBX in software, as well as l2tpd and the Cheops Network User Interface.\\
\\
\textbf{\underline{Theo De Raadt:}}\\
Software engineer, known to be the founder and leader of the OpenBSD and OpenSSH projects, and founding member of the NetBSD project.\\
\\
\textbf{\underline{Michael Tiemann:}}\\
Contributor to free software include authorship of the GNU C++ compiler and work on the GNU C compiler and the GNU Debugger.\\
\\
\textbf{\underline{Spencer Kimball and Peter Mattis:}}\\
They both started development of Gimp in 1995 on Berkeley University, California.\\
\newpage
{\noindent}\textbf{\underline{Bill Joy / Chuck Haley:}}\\
The original code for vi text editor was written by Bill Joy in 1976, as the visual mode for a line editor called ex that Joy had written with Chuck Haley.\\ 
\\
\textbf{\underline{Theodore Y. "Ted" Ts'o:}}\\
Software developer mainly known for his contributions to the Linux kernel, in particular his contributions to file systems (e2fsprogs, the userspace utilities for the ext2 and ext3 filesystems, and maintainer for the ext4 file system).\\
\\
\textbf{\underline{Harald Welte:}}\\
German programmer, well known as a hacker of the Linux kernel, enforcer of the GNU General Public License (GPL).  Also involved in a number of free software projects, such as Openmoko, (a version of Linux for completely open, low-cost, high-volume phones) and the netfilter/iptables project.\\
\\
\textbf{\underline{Jörg Schilling:}}\\
Computer programmer who has worked extensively on compact disc burning software "cdrtools", the Solaris Operating System and the OpenSolaris project. He has also been involved on BerliOS Linux distribution.\\
\\
\textbf{\underline{Jamie Zawinsky:}}\\
American Computer programmer responsible for significant contributions to the free software projects Mozilla and XEmacs, and early versions of the Netscape Navigator web browser. He also maintained the XScreenSaver project.\\
\\
\textbf{\underline{Rob Savoye:}}\\
Primary developer of Gnash, he is developer of the GNU Project. He has also worked on Red Hat, Debian, and other libre software projects.\\
Some of the projects he has worked on include GCC, GDB, DejaGnu, Cygwin, eCos and CTAS.\\
\\
\textbf{\underline{Árpád Gereöffy:}}\\ 
Founder of MPlayer in 2000 and main developer in the early beginning of this audio and video player application.\\
\\
\textbf{\underline{Fabrice Beillard:}}\\
Computer programmer who is best known as the creator of the FFmpeg and QEMU software projects.\\
\\
\textbf{\underline{Dries Buytaert:}}\\
Open-source software programmer notable as founder and lead developer of the Drupal CMS.\\
\\
\textbf{\underline{Ton Roosendaal:}}\\
Dutch software developer, known as the original creator of the open-source 3D creation suite Blender.\\
\\
\textbf{\underline{Andrew Tridgell:}}\\
Australian computer programmer Author of the Samba file server and co-inventor of the rsync algorithm.\\
\textbf{\underline{Brian E. Paul:}}\\
Computer programmer who originally wrote and continues to maintain the source code for the open source Mesa graphics library.\\
\\
\textbf{\underline{John Gilmore:}}\\
Worked on several GNU projects, including maintaining the GNU Debugger in the early 90s, initiating GNU Radio in 1998, starting Gnash in December 2005 to create a free software player for Flash movies, and writing the pdtar program which became GNU tar.\\
\\
\textbf{\underline{Wietse Zweitze Venema:}}\\
Dutch programmer best known for writing the Postfix email system.\\
\\
\textbf{\underline{Marco Pesenti:}}\\
Main developer of Galeon, Gnome's Browser.\\
\\
\textbf{\underline{Guenter Bartsch:}}\\
Creator of Xine Media Player.\\
\\
\textbf{\underline{Mark Kretschmann:}}\\
Creator of the Amarok audio player.\\
\\
\textbf{\underline{Roger Dannemberg / Dominic Mazzoni:}}\\
Creators of the Audacity free digital audio editor.\\
\\
\textbf{\underline{Olivier Fourdan:}}\\
Creator and main developer of the XFCE desktop.\\

\section* {Programming Languages}

Apart from Software applications tools, Programming Languages deserve special mention. Some of the most powerful contributors to libre software are creators of the most world widely used programming languages:\\
\\
\textbf{\underline{Larry Wall:}}\\
Creator of Perl Programming Language, also known for beeing the developer of the "patch" tool or "rn", the Usenet client.\\
\\
\textbf{\underline{Guido Van Rossum:}}\\
Dutch computer programmer, author of the Python programming language.\\
\\
\textbf{\underline{Yukihiro Matsumoto:}}\\
Japanese computer scientist and software programmer, best known as the chief designer of the Ruby programming language.\\
\\
\textbf{\underline{Brendan Eich:}}\\
Computer programmer and creator of the JavaScript scripting language. He is the chief technology officer at the Mozilla Corporation.\\
\\

\section* {Linux Distributions}

Not all in life is software. Linux Distributions, as well, have contributed to compilate complete Operating Systems for the final user.
Regarding Linux Distributions, next names should be considered as great contributors to the Libre Software philosophy:\\
\\
\textbf{\underline{Ian Murdock:}}\\
Founder of the Debian distribution, wrote the Debian Manifesto in 1993 while a student at Purdue University.\\
\\
\textbf{\underline{Daniel Robbins:}}\\
Founder and former chief architect of the Gentoo Linux project.\\
\\
\textbf{\underline{Rob Young:}}\\
Robert "Bob" Young is a serial entrepreneur whose biggest success has been Red Hat Inc, the open source software company.\\
\\
\textbf{\underline{Gaël Duval:}}\\
Graduate of the Caen University in France. In July 1998, he created Mandrake Linux (now Mandriva Linux), a Linux distribution originally based on Red Hat Linux and KDE.\\

\section* {Closure}

Free Software / Open Source / Libre Software as general concept, is a wide concept \textbf{contributed by dozens and dozens of thousands of not only software developers}, but also \textbf{software distribution packagers, activists, translators, documentalists and other roles} that have cooperated to establish one of the most important advances on the last fifty years.\\
This text is a recognition \textbf{not only for the less known characters, but also for the anonymous people involved}, and aims to \textbf{cooperate with Libre Software for each individual} as far as possible, for all the job done not to be dissipated, but, instead, to continue being improved for the user freedom.

\end{document}
