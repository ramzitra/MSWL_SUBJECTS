\documentclass[11pt]{article}
%Gummi|061|=)
\title{\textbf{MSWL Developers and Motivations: Linus Torvalds}}
\author{Sergio Arroutbi Braojos}
\usepackage{hyperref}
\date{\today}
\usepackage[bottom=14em]{geometry}
\usepackage{amsmath}
\usepackage{mathtools}

\begin{document}

\hypersetup
{   
pdfborder={0 0 0}
}
   
\maketitle

\tableofcontents

\pagebreak

\section{Introduction}
Linus Torvalds is, surely, the most influencing person on the Open Source World. Twenty years after initiation of the Linux project by him, this open source Operating System being the main maintainer of the Linux project, this Operating System has become, surely, one of the most used and versatile Operating Systems, leader in some markets, as the Internet Servers Operating Systems or the SuperComputers Operating Systems, as well as the most important Open Source Project, at least, from an economic impact point view.\\
\\
Linus Torvalds is also creator and developer of the GIT source code control version system, the most used distributed version control system, with an increasing popularity above all on the Open Source projects enviroment. However, Torvalds turned over maintenance on 26 July 2005 to Junio Hamano, a major contributor to the project, so GIT is not a relevant aspect of the analysis of Linus Torvalds as Open Source Project Manager, compared to up to date leadership on the Linux Kernel Operating System project.

\section{Biography}

\section{The Linux Project}
\subsection{Project success}
\subsection{Project licensing}

\section{Linux Project Management}
\subsection{Project Architecture}
\subsection{Project Release System}
\subsection{Project Roles}
\subsection{Linus Torvalds: The (not so?) benevolent dictator}

\section{Conclussion}

\end{document}
